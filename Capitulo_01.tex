\chapter{\label{cap:1} Nanonomateriales}

\noindent
\rule{\linewidth}{1 pt}
\begin{flushright}
\begin{quotation}
\small{
\textit{``There’s Plenty of Room at the Bottom.''}}
\end{quotation}
\bf{Richard Feymann}
\end{flushright}
\noindent
\rule{\linewidth}{1 pt}\\
\vspace{1cm}


\section{Nanomateriales}
En un nanomaterial\index{nanomaterial} al menos una de sus dimensiones está en la escala nanométrica. En una mejor definición, hablamos de nanomateriales cuándo alguna de sus dimensiones es menor a alguna de sus longitudes características\index{longitud característica}, dando lugar a la aparición de propiedades diferentes a las de su contraparte macrométrica (bulk material). Los nanomateriales pueden clasificarse por el número de dimensiones en escala nanométrica, con una dimensión constreñida a nanoescala hablamos de materiales 2-dimensionales, pues dos dimensiones están en la macroescala, análogamente, con dos a nanoescala tenemos un material 1-dimensional, y con tres dimensiones a nanoescala, es un material 0-dimensional. Ejemplos de nanomateriales: quantum dots,  nanopartículas (0-dimensional); nanotubos, nanohilos, nanovarillas (1-dimensional); grafeno (2-dimensional).

\section{Síntesis}


\section{Caracterización}

\section{Aplicaciones}
