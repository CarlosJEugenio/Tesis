%Chapter{Sintesis de óxido de grafeno}
El método de síntesis utilizado está basado en el propuesto por Hummers para la síntesis de óxido de grafito \citep{Hummers1958}, y es descrito por Abdolhosseinzadeh \citep{Abdolhosseinzadeh2015}.

\section{Procedimiento experimental}
En una síntesis normal 3 g de grafito en polvo (Sigma-Aldrich $>$99\%) o en hojuelas (Superior Graphite $>$80\%), tal como viene envasado, se añaden a 150 ml de ácido sulfúrico (\ce{H_2SO_4}, Baker 97.8\%) en un vaso precipitado de 1000 ml previamente puesto en un baño de hielo sobre un agitador magnético, la mezcla se deja agitar por 5 minutos. Una vez el grafito se ha dispersado en el ácido sulfúrico se agregan 9 g de permanganato de potasio (\ce{KMnO_4}, Chemix 99.44\%), lentamente, manteniendo la temperatura de la mezcla bajo los 10 C y así evitar la explosión del permanganato. Luego de 15 minutos de agitación, se quita del baño de hielo y se agita 25 minutos más a temperatura ambiente, seguido de 5 minutos en un baño ultrasónico (99\% de potencia, SB-3200DTD Ultrasonic Cleaner), este proceso de agitación-sonicación es repetido 12 veces, tomando un total de 6 horas en completarse. Una vez completado este proceso, se agregan 600 ml de agua destilada rápidamente, esto produce una reacción exotérmica con evolución de gases, la solución se vuelve marrón, color característico del óxido de grafeno. En general, la solución se divide en dos partes iguales, y se vuelven a sonicar por dos horas más, una de las partes es reducida posteriormente, mientras que la otra es trata con agua oxigenada (\ce{H_2 O_2}) tal como lo menciona Hummers en su escrito original, para reducir el permanganato y dióxido de manganato restante \citep{Hummers1958}, 60 ml de \ce{H_2_O2} al 30\% se añaden lentamente mientras la se agita, la solución se vuelve de color amarillo brillante y se deja precipitar a temperatura ambiente, para luego lavarla con agua destilada varias veces.

\section{Resultados}