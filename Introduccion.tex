%%%INTRODUCCIÓN GENERAL%%%
\chapter*{Introducción}
\addcontentsline{toc}{chapter}{Introducción}

Esta introducción trata sobre que es la nanotecnología
El prefijo nano deriva del griego \emph{nanos}, que significa literalmente ``enano''. En el sistema internacional de unidades, el prefijo nano representa un factor de $\mathrm{10^{-9}}$, o una mil millonésima. Al añadir el prefijo a la unidad de longitud, obtenemos ``nanómetro'' (nm), o una mil millonésima parte de un metro. Así la nanotecnología se define como la ciencia, tecnología, e ingeniería que trata sistemas en el rango aproximado de 1-100 nanómetros \citep{Haick2013}.
La idea de la nanotecnología fue vislumbrada por el físico Richard Feynman y expuesta en su charla \emph{``There is plenty room at the bottom''} \citep{Feynman1960}. Aquí Feynman plantea que no existen barreras físicas que impidan manipular sistemas nanométricos, moléculas, o átomos.

\begin{figure}
	\centering
	\includegraphics[width=\textwidth,draft]{dimension_scale.png}
	\caption{Dimensiones}
	\label{fig:scale}
\end{figure}

\section*{La física de sistemas nanométricos}

