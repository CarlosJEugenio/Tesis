%\chapter{Nanomateriales} //is included in Marco teorico file
\noindent
\rule{\linewidth}{1 pt}
\begin{flushright}
	\begin{quotation}
		\small{
			\textit{``There’s Plenty of Room at the Bottom.''}}
	\end{quotation}
	\bf{Richard Feymann}
\end{flushright}
\noindent
\rule{\linewidth}{1 pt}\\
\vfill
\section{Nanomateriales}
Generalmente la denominación nano es atribuida a materiales en que algunas de sus dimensiones estén en la escala nanométrica, entre 1-100 nm \cite{Gressler2013}. Ésta definición resulta práctica para fines 
\section{Síntesis}

\section{Caracterización}

\section{Aplicaciones}
