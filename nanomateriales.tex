%\chapter{Nanomateriales} //is included in Marco teorico file
\noindent
\rule{\linewidth}{1 pt}
\begin{flushright}
	\begin{quotation}
		\small{
			\textit{``There’s Plenty of Room at the Bottom.''}}
	\end{quotation}
	\bf{Richard Feymann}
\end{flushright}
\noindent
\rule{\linewidth}{1 pt}\\
\vfill
\section{Nanomateriales}
Generalmente la denominación nano es atribuida a materiales en que algunas de sus dimensiones estén en la escala nanométrica, entre 1-100 nm \cite{Gressler2013}. Ésta definición es práctica pero poco precisa en el sentido que algunos materiales exhiben propiedas nuevas fuera de este rango (> 100 nm). Por esta razón es preferible otorgar el sufijo nano a un material en cual alguna de sus dimensiones es equiparable a alguna longidud característica de este.
\section{Síntesis}
Existen dos formas de sintetizar nanomateriales dependiendo de la vía de aproximación a la nanoescala, por un lado, si partimos la forma macro de un material y de algún modo se reducen sus dimensiones hacia la nanoescala, se habla de un proceso \textit{top-down}. Por ejemplo, la exfliación del grafito (macro) para obtener grafeno (nano).  Por otro lado, sintetizar un nanomaterial a partir de átomos o moléculas es un proceso \textit{bottom-up}, un ejemplo, es la síntesis de nanopartículas de oro a partir de un precursor como el ácido tetracloroaúrico.

\section{Caracterización}

\section{Aplicaciones}
