%\chapter{Nanomateriales} //is included in Marco teorico file
\noindent
\rule{\linewidth}{1 pt}
\begin{flushright}
	\begin{quotation}
		\small{
			\textit{``There’s Plenty of Room at the Bottom.''}}
	\end{quotation}
	\bf{Richard Feymann}
\end{flushright}
\noindent
\rule{\linewidth}{1 pt}\\
\vfill
\section{Nanomateriales}
%TODO ejemplos de longitudes características
Generalmente la denominación nano es atribuida a materiales en que algunas de sus dimensiones estén en la escala nanométrica, entre 1-100 nm \citep{Gressler2013}. Ésta definición es práctica pero poco precisa en el sentido que algunos materiales exhiben características propias de los nanomateriales fuera de este rango ($>$ 100 nm). Por esta razón es preferible hablar de nanomateriales cuando se comienzan a mostrar éstas nuevas características. El momento en cual aparecen estos cambios, es propia de cada material, y está asociado a alguna longitud característica de éste. Algunos ejemplos, el camino libre medio de un electrón,

\section{Síntesis}
Dependiendo de la vía de aproximación a la nanoescala, se distinguen dos formas de síntesis, por un lado, si partimos de la forma macro de un material y de algún modo se reducen sus dimensiones hacia la nanoescala, se habla de un proceso \textit{top-down}. Por ejemplo, la exfoliación del grafito (\textit{bulk material}) para obtener grafeno (nanomaterial) \citep{Novoselov2004}.  Por otro lado, sintetizar un nanomaterial a partir de átomos o moléculas es un proceso \textit{bottom-up}, un ejemplo es la síntesis de nanopartículas de oro a partir de un precursor como el ácido tetracloroaúrico \citep{Daniel2004}.

\section{Caracterización}


\section{Aplicaciones}

