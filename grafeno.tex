%\chapter{Grafeno}
\noindent
\rule{\linewidth}{1 pt}
\begin{flushright}
	\begin{quotation}
		\small{
			\textit{``What is important about graphene is the new physics it has delivered.''}}
	\end{quotation}
	\bf{Andre Geim}
\end{flushright}
\noindent
\rule{\linewidth}{1 pt}\\
\vfill

%TODO add graphene before 2004
El grafeno es un nanomaterial 2-dimensional, formados por átomos de carbono  hibridizados sp2, en una estructura de panal de abeja, descrito como dos redes hexagonales superpuestas. El primero en tratar con este material fue probablemente \citet{Brodie1859} que al exponer grafito a ácidos fuertes, creyó descubrir una nueva forma de carbono a la que llamó grafón, ahora se sabe que lo que observó fue óxido de grafeno, esto es, laminas de grafeno recubiertas por grupos epóxi e hidroxilo \citep{Geim2012}. \citet{Wallace1947}, dio con los primeros estudios teóricos sobre el grafeno al estudiar la estructura de bandas del grafito, pero como una simplificación de la estructura del grafito. Fue \citet{Boehm1986} quien le dio el nombre al crear la nomenclatura y terminología para compuestos de grafito intercalado. No fue hasta 2004 que Novoselov, Geim y otros, lograron aislar grafeno por medios mecánicos \citep{Novoselov2004}, por lo que les fue otorgado el Premio Nobel de Física en 2010.

El grafeno presenta propiedades extraordinarias, como ha sido demostrado en numerosos experimentos: movilidad electrónica de 200.000 $\mathrm{cm^2 V^{-1} s^{-1} }$\citep{Bolotin2008}, tensión de ruptura de 130 GPa, y módulo de Young de 1.0 TPa \citep{Lee2008}, conductividad térmica entre 600 a 5000 $\mathrm{W mK^{-1}}$ \citep{Balandin2011}, opacidad de 2,3\% y reflectacia menor al 0,1\% \citep{Nair2008}.

\section{Óxido reducido de grafeno (RGO)}
Para producir grandes cantidades de grafeno, con el objetivo de utilizarlo en dispositivos que requieran una cantidad considerable de este, como aditivo en materiales compuestos, baterías, supercondensadores, etc. se hace necesario buscar nuevos métodos que entreguen suficiente material utilizable, y a su vez, de bajo costo, e idealmente amigable con el medio ambiente.
Una de las formas de obtener grandes cantidades de grafeno es mediante la llamada ruta del "óxido de grafeno".

\section{Síntesis de RGO}
El precursor con más disponibilidad y menor precio, es el grafito en estado natural que se obtiene directamente de la mina.

Para producir grafeno que sea utilizable en aplicaciones que requieran grandes cantidades de este, como recubrimientos, baterías, supercondensadores,etc. se hace necesario buscar nuevos métodos que entreguen volúmenes grandes de material utilizable, y a su vez, de bajo costo, e idealmente amigable con el medio ambiente. La vía que puede cumplir con estos requisitos, es mediante la oxidación del grafito para exfoliarlo, y su posterior reducción.

\section{Síntesis}


\section{Aplicaciones}