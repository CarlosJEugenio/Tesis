%\chapter{Grafeno}
\noindent
\rule{\linewidth}{1 pt}
\begin{flushright}
	\begin{quotation}
		\small{
			\textit{``What is important about graphene is the new physics it has delivered.''}}
	\end{quotation}
	\bf{Andre Geim}
\end{flushright}
\noindent
\rule{\linewidth}{1 pt}\\
\vfill

El grafeno es un nanomaterial 2-dimensional, formados por átomos de carbono  hibridizados sp2, en una estructura de panal de abeja. El primero tratar con este material fue probablemente Brodie en 1859. Al exponer grafito a ácidos fuertes, creyó descubrir una nueva forma de carbono a la que llamó grafon, ahora se sabe que lo que observó fue óxido de grafeno, esto es, laminas de grafeno recubiertas por grupos epóxi e hidroxilo\cite{Geim2012}. Wallace en 1947, dio con los primeros estudios teóricos sobre el grafeno al estudiar la estructura de bandas del grafito \cite{Wallace1947}.

\section{Óxido reducido de grafeno (RGO)}
Para producir grandes cantidades de grafeno, con el objetivo de utilizarlo en dispositivos que requieran una cantidad considerable de este, como aditivo en materiales compuestos, baterías, supercondensadores, etc. se hace necesario buscar nuevos métodos que entreguen suficiente material utilizable, y a su vez, de bajo costo, e idealmente amigable con el medio ambiente.
Una de las formas de obtener grandes cantidades de grafeno es mediante la llamada ruta del "óxido de grafeno".

\section{Síntesis de RGO}
El precursor con más disponibilidad y menor precio, es el grafito en estado natural que se obtiene directamente de la mina.

Para producir grafeno que sea utilizable en aplicaciones que requieran grandes cantidades de este, como recubrimientos, baterías, supercondensadores,etc. se hace necesario buscar nuevos métodos que entreguen volúmenes grandes de material utilizable, y a su vez, de bajo costo, e idealmente amigable con el medio ambiente. La vía que puede cumplir con estos requisitos, es mediante la oxidación del grafito para exfoliarlo, y su posterior reducción.

\section{Síntesis}


\section{Aplicaciones}