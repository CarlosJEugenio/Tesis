%\chapter{Grafeno}
\noindent
\rule{\linewidth}{1 pt}
\begin{flushright}
	\begin{quotation}
		\small{
			\textit{``What is important about graphene is the new physics it has delivered.''}}
	\end{quotation}
	\bf{Andre Geim}
\end{flushright}
\noindent
\rule{\linewidth}{1 pt}\\
\vfill

\section{Óxido reducido de grafeno (RGO)}
<<<<<<< HEAD
Para producir grandes cantidades de grafeno, con el objetivo de utilizarlo en dispositivos que requieran una cantidad considerable de este, como recubrimientos, baterías, supercondensadores,etc. se hace necesario buscar nuevos métodos que entreguen grandes cantidades de material utilizable, y a su vez, de bajo costo, e idealmente amigable con el medio ambiente. 
\section{Síntesis de RGO}
El precursor con más disponibilidad y menor precio, es el grafito en estado natural, 
=======
Para producir grafeno que sea utilizable en aplicaciones que requieran grandes cantidades de este, como recubrimientos, baterías, supercondensadores,etc. se hace necesario buscar nuevos métodos que entreguen volúmenes grandes de material utilizable, y a su vez, de bajo costo, e idealmente amigable con el medio ambiente. La vía que puede cumplir con estos requisitos, es mediante la oxidación del grafito para exfoliarlo, y su posterior reducción.

\section{Síntesis}
>>>>>>> af94729f118c7e33e4442a07399a5ed849acfdaa


\section{Aplicaciones}