%\chapter{Grafeno}
%TODO add graphene before 2004
%TODO add graphene synthesis
El grafeno es un nanomaterial 2-dimensional, formados por átomos de carbono  hibridizados sp2, en una estructura de panal de abeja, esta estructura no es una red de Baravais pero puede ser descrita por una red hexagonal con una base de dos átomos de carbono. El primero en tratar con este material fue probablemente Brodie \citep{Brodie1859} que al exponer grafito a ácidos fuertes, creyó descubrir una nueva forma de carbono a la que llamó grafón, ahora se sabe que lo que observó fue óxido de grafeno, esto es, laminas de grafeno recubiertas por grupos epóxi e hidroxilo \citep{Geim2012}. Wallace, dio con los primeros estudios teóricos sobre el grafeno al estudiar la estructura de bandas del grafito, pero como una simplificación de la estructura del grafito \citep{Wallace1947}. Fue Boehm quien le dio el nombre al crear la nomenclatura y terminología para compuestos de grafito intercalado \citep{Boehm1986}. No fue hasta 2004 que Novoselov, Geim y otros, lograron aislar grafeno por medios mecánicos \citep{Novoselov2004}, por lo que les fue otorgado el Premio Nobel de Física en 2010.

El grafeno presenta propiedades extraordinarias, como ha sido demostrado en numerosos experimentos: movilidad electrónica de 200.000 $\mathrm{cm^2 V^{-1} s^{-1} }$\citep{Bolotin2008}, tensión de ruptura de 130 GPa, y módulo de Young de 1.0 TPa \citep{Lee2008}, conductividad térmica entre 600 a 5000 $\mathrm{W\, mK^{-1}}$ \citep{Balandin2011}, opacidad de 2,3\% y reflectacia menor al 0,1\% \citep{Nair2008}, impermeable totalmente a gases estándar \citep{Bunch2007}, resistir densidades de corriente muy grandes de hasta $\mathrm{10^8 A\, cm^{-2}}$ sin sufrir daños  \citep{Moser2007}, y puede ser funcionalizado fácilmente \citep{Loh2010}. Es importante notar que estos resultados se han obtenido en muestras muy puras de grafeno exfoliado mecánicamente \citep{Novoselov2004} y están lejos de ser replicables a gran escala, se hace necesario encontrar métodos de síntesis que entreguen material de buena calidad (que sus propiedades se acerquen a las citadas anteriormente), y sean escalables a niveles industriales \citep{Novoselov2012}.

\section{Óxido reducido de grafeno (rGO)}
%TODO graphene oxide properties
Una de las formas de obtener grandes cantidades de grafeno es mediante la llamada ruta del "óxido de grafeno". El óxido de grafeno es grafeno decorado densamente por grupos epóxi, hidroxilo, y carboxilo \citep{Dreyer2010}. Los grupos ricos en oxígeno presentes en la red del grafeno se presentan como defectos en éste, cambiando sus propiedades drásticamente.
El óxido de grafeno es sintetizado a partir de grafito natural, exponiéndolo a agentes oxidantes fuertes, esto introduce grupos funcionales ricos en oxígeno en los espacios entre los planos de grafeno del grafito, aumentado la distancia interplanar, y disminuyendo la fuerza entre láminas. Esto facilita la separación de las láminas de grafeno (ahora óxido de grafeno). La mayoría de los métodos de síntesis del óxido de grafeno están basados en alguno de estos tres métodos: método de Brodie \citep{Brodie1859}, método de Staudenmaier \citep{Staudenmaier1898}, o método de Hummers \citep{Hummers1958}.



\section{Síntesis de rGO}
Los grupos funcionales presentes en el óxido de grafeno pueden ser removidos para volver a la estructura del grafeno como tal. Existen muchas formas de reducir el óxido de grafeno, por medios químicos, térmicos, o electroquímicos. En los métodos de reducción química, el óxido de grafeno es expuestos a diferentes agentes reductores, cuyo mecanismo de reducción es sabido, se prueba esperando el efecto deseado \citep{Chua2015}. Por otro lado, la reducción térmica contempla la exposición del óxido de grafeno a altas temperaturas en un horno convencional, reducción por microondas en un horno microondas comercial \citep{Zhu2010a}, reducción por láser \citep{El-Kady2013}, plasma \citep{Lee2012}, o luz solar concentrada \citep{Mohandoss2017}. La reducción electroquímica se realiza en presencia de un solvente, el óxido de grafeno puede estar disperso en el solvente \citep{Liu2011}, depositado en un electrodo \citep{Harima2011, Toh2014}, o bien actuar como electrodo por sí mismo \citep{Feng2016}. Una gran ventaja de los métodos electroquímicos es la facilidad de realizar electrodeposición del rGO en otro electrodo y la combinación con otros nanomateriales, por ejemplo, en una síntesis \emph{in-situ} \citep{Liu2011, Xie2014}.

\section{Aplicaciones}