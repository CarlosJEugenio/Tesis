%\chapter{Construcción de supercondensadores}
Un supercondensadores es armado simplemente haciendo un sándwich electrodo-separador-electrodo, el electrodo puede ser una lámina de material o estar depositado en un sustrato metálico, cualquiera sea el caso, el sándwich es introduce en la celda de prueba (Figura \ref{fig:celda_de_pruebas_SC}). 

\section{Celda de pruebas de supercondensador}
Se diseña una celda para realizar las pruebas de supercondensadores con los materiales sintetizados. La celda (Figura \ref{fig:celda_de_pruebas_SC}) consta de dos colectores de corriente de acero inoxidable, entre los que se ubica el condensador como tal. Los colectores de corriente tienen sellos que impiden la fuga del electrolíto o la evaporación del agua en él, permitiendo una operación estable en el tiempo. Los colectores de corriente se apoyan en bloques de acero que cierran la celda con pernos y permitan conectar los terminales de potenciómetro a la celda

\begin{figure}[h!]
	\centering
	\fbox{
		\includegraphics[width=0.8\textwidth]{cell2.png}
		}
	\label{fig:celda_de_pruebas_SC}
	\caption{}
\end{figure}


\section{Resultados}
Los supercondensadores son sometidos a pruebas electroquímicas para estudiar su desempeño, estás pruebas incluyen: voltametría cíclica (CV), ciclos de carga y descarga a corriente constante, espectroscopía de impedancia electroquímica (EIS).
