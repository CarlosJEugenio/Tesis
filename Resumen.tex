%%RESUMEN
\chapter*{Resumen}
El grafeno es un nanomaterial que encuentra su lugar en un ancho espectro de aplicaciones. En este trabajo de tesis se explora su potencial como material activo en electrodos para supercondensadores.\\
Con este objetivo, se sintetiza óxido de grafeno reducido (rGO), material cuyas propiedades físicas se acercan a las del grafeno como tal. La vía de síntesis empleada comienza con grafito natural en hojuelas, que a causa de un agente oxidante fuerte, se convierte en óxido de grafeno. En este caso, el agente oxidante utilizado es heptaóxido de manganeso (\ce{MnO_7}). Luego, éste es reducido por medio de ácido ascórbico (\ce{C_6H_8O_6}), dando como producto final óxido de grafeno reducido.\\
Este último es utilizado como material activo en la fabricación de electrodos de supercondensadores, empleando diferentes sustratos metálicos y métodos de deposición. Para caracterizar el desempeño de tales electrodos, se diseña y construye una celda de prueba, que en conjunto con la metodología de medición, resulta ser apropiada para estudiar diferencias entre los métodos de fabricación de electrodos.\\
Se concluye que el mejor desempeño se consigue utilizando óxido reducido de grafeno liofilizado y depositado en sustratos metálicos no porosos. Sin embargo, considerando el costo de este proceso, se concluye que la mejor forma de preparar electrodos, es usando láminas similares al papel fabricadas mediante filtración al vacío.\\
\hfill
\textbf{Palabras claves:} Supercondensadores, óxido reducido de grafeno, nanomateriales.
