\chapter{Conclusiones}
%\addcontentsline{toc}{chapter}{Conclusiones}
En este trabajo de tesis se sintetizó un material de carbono para ser utilizado como electrodo en supercondensadores, para lo que se diseñó y construyó una celda de pruebas, que junto con una metodología de medición, resultó ser apropiada para estudiar las diferencias en la preparación de electrodos de supercondensadores.

Los electrodos de papel y liofilizados en sustrato de acero (\mPapelAcero y \mLiofilizadoAcero), mostraron tener mejor desempeño en todas las pruebas a las que fueron sometidos, mayor capacitancia específica en la voltametría cíclica, menor resistencia en serie equivalente en los ciclos de carga y descarga, y mejor comportamiento capacitivo en la espectroscopia de impedancia electroquímica.

Por otro lado, los electrodos 

La principal fuente de error en la caracterización electroquímica es el error instrumental de la balanza en la medición de la masa de material activo presente al ensamblar la celda de pruebas de supercondensador. La causa de este error es la minúscula cantidad de masa depositada en los electrodos, siendo cercana al límite de sensibilidad de la balanza. Este error se reduce aumentando la sensibilidad de la balanza, o bien aumentando la masa de material activo de los electrodos. Esto último es ha de tratarse con especial cuidado, pues al aumentar la masa, puede quedar material sin entrar en contacto con el electrolito, disminuyendo la capacidad específica calculada. 