\chapter{Conclusiones}
%\addcontentsline{toc}{chapter}{Conclusiones}
En este trabajo de tesis se implementó un método efectivo para medir el desempeño de materiales basados en carbono como electrodos de supercondensadores, que resultó ser apropiado para determinar diferencias en la construcción de los electrodos. Además, se sintetizó un material de carbono para ser utilizado como electrodo en dichos condensadores.

Comparando la capacidad específica medida para los electrodos de polvo en sustrato de acero (\mPolvoAcero), y el de polvo en espuma de níquel (\mPolvoNiquel), se observa que el sustrato de acero se desempeña mejor que el de níquel, a pesar de que el sustrato de níquel posee mayor area superficial.

El gráfico de Nyquist revela que los electrodos que exhiben un comportamiento más cercano al esperado para un supercondensador, son las 
muestras de papel y liofilizado.
